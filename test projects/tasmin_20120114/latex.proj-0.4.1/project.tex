% 
% This document is copyrighted by Robert G. Brown as of the latest 
% revision date below, as are all earlier revisions.
%
% $Id: project.tex 37 2006-10-26 06:54:23Z rgb $
%
% Filename: project.tex
% Description:  Latex sources for a simple article/project

\documentclass{article}
\usepackage{epsfig}
\usepackage{html}

% Useful parameters to set in the book documentclass.
% \setlength{\topmargin}{-1.0in}
% \setlength{\textheight}{9.2in}
% \setlength{\oddsidemargin}{0.0in}
% \setlength{\textwidth}{6.5in}
%   EITHER comment both out OR uncomment both of these, usually.
%   As it is, it doesn't indent but skips a small space between
%   paragraphs, good for e.g. lecture notes, not so good for books
%   or articles depending.
\setlength{\parindent}{0.0in}
\setlength{\parskip}{0.1in}

\newcommand{\martin}[1]{\textcolor{blue}{\textit{Martin: #1}}}
\newcommand{\miquel}[1]{\textcolor{red}{\textit{Miquel: #1}}}
\newcommand{\christian}[1]{\textcolor{green}{\textit{Ch.: #1}}}

\newcommand{\etc}{\textit{etc}}
\newcommand{\ie}{\textit{i.e.}}
\newcommand{\da}{\textsc{LTPDA}}
\newcommand{\code}[1]{\texttt{#1}}
\newcommand{\tbd}[0]{\textcolor{red}{TBD}}
\newcommand{\ltpda}[0]{\textsc{LTPDA}}

\newcommand{\note}[1]{\textcolor{blue}{(#1)}}




% Spare figure template
%\begin{figure}
%  \centerline{
%    \def \epsfsize#1#2{0.5#1}
%    \centerline{\epsfbox{vector.1.eps}}
%  }
%  \caption{This would be a figure.}
%  \label{figure_label}
%\end{figure}
%
% Inline unnumbered figure (no caption)
% \def \epsfsize#1#2{0.5#1}
% \centerline{\epsfbox{vector.1.eps}}
% \medskip
% 
%

\begin{document}

\title{A Nifty Latex Template}

\date{\today}

\author{\bf Robert G. Brown \\
Duke University Physics Department \\
Durham, NC 27708-0305 \\
rgb@phy.duke.edu}

\maketitle

\vspace*{\fill}

\centerline{\large \bf Copyright Notice}
\centerline{Copyright Robert G. Brown 2006}
\centerline{See attached Open Publication License}

\newpage

\tableofcontents

\newpage

% Inline figure example
\def \epsfsize#1#2{0.5#1}
\centerline{\epsfbox{project.eps}}

% For all books and most complex documents (with many sections) it is
% probably a good idea to break up the source into many smaller files, one
% per chapter or section.  You could then input them like:
% \input{section_one.tex} or \input{chapter_one.tex}
% This document is very simple, so I don't bother, except for the OPL
% which is "standard" to all my documents of this sort.  If you don't
% like it or need it, comment it out or delete it and remove the file.
\section{Section One}

This is text in section one.

It can contain inline math: $E = \hbar \omega$.

It can contain numbered equations:
\be
  H\psi = E\psi = \hbar \omega \psi
\ee

It can even contain numbered aligned equation ``arrays'':
\bea
 H\psi & = & -i\hbar \partialdiv{\psi}{t} \nonumber \\
       & = & \hbar \omega \psi \nonumber \\
       & = & E \psi
\eea


\subsection{Subsectioning Text}

Subsectioning text is certainly possible, and yields a trace in the
table of contents (if any).

\subsubsection{Subsubsectioned Text}

Text at this level still gets a TOC entry in some document classes, but
not all.  TOC's usually only go three levels deep, four if you use
the ``part'' command in a book.  Still useful for the formatting.

That's {\em really} about it.  You can do stuff like {\bf boldfacing}
and using {\tiny tiny} fonts or {\huge HUGE} fonts, but you probably
shouldn't as they make the text look odd.  You can do tables and tabs
and much more, but that is too much to demo in a simple template like
this, especially when a lot of that will vary as you add packages.

Simple latex is the best, unless you are writing for a very specific
purpose with very specific requirements.  It is what latex is really
designed for -- the whole idea of markup is to trust {\em professionals}
to lay out appropriate fonts, sizes, and so on for various document
objects in a completely uniform way.  Latex documents invariably look
like they are ripped right out of the pages of a book, even when they
are really simple ones (like this one)!

Good Luck!

The following appendix is an {\em example} of a Gnu Open Publication
License.  Don't worry, you can take this template and use it any way
that you wish, with or without the OPL announcement.  However, if you
use it you owe me a beer...

\newpage
\appendix
\section{License Terms for ``Latex Project Template''}

\subsection{General Terms}

License is granted to copy or use this document according to the Open
Public License (OPL, enclosed below), which is a Public License,
developed by the GNU Foundation, which applies to ``open source''
generic documents.

In addition there are three modifications to the OPL:

Distribution of substantively modified versions of this document is
prohibited without the explicit permission of the copyright holder.
(This is to prevent errors from being introduced which would reflect
badly on the author's professional abilities.)

Distribution of the work or derivative of the work in any standard
(paper) book form is prohibited unless prior permission is obtained from
the copyright holder. (This is so that the author can make at least some
money if this work is republished as a textbook or set of notes and sold
commercially for -- somebody's -- profit.  The author doesn't care about
copies photocopied or locally printed and distributed free or at cost to
students to support a course, except as far as the next clause is
concerned.)

The "Beverage" modification listed below applies to all non-Duke usage
of these notes in any form (online or in a paper publication).  Note
that this modification is probably not legally defensible and can be
followed really pretty much according to the honor rule.

As to my personal preferences in beverages, red wine is great, beer is
delightful, and Coca Cola or coffee or tea or even milk acceptable to
those who for religious or personal reasons wish to avoid stressing my
liver.  Students at Duke, whether in my class or not, of course, are
automatically exempt from the beverage modification.  It can be presumed
that the fraction of their tuition that goes to pay my salary counts for
any number of beverages.

\bigskip

\subsection{The ``Beverage'' Modification to the OPL}

Any user of this OPL-copyrighted material shall, upon meeting the
primary author(s) of this OPL-copyrighted material for the first time
under the appropriate circumstances, offer to buy him or her or them a
beverage.  This beverage may or may not be alcoholic, depending on the
personal ethical and moral views of the offerer(s) and receiver(s).  The
beverage cost need not exceed one U.S. dollar (although it certainly may
at the whim of the offerer:-) and may be accepted or declined with no
further obligation on the part of the offerer.  It is not necessary to
repeat the offer after the first meeting, but it can't hurt...

\bigskip

\subsection{OPEN PUBLICATION LICENSE Draft v0.4, 8 June 1999}

{\bf I. REQUIREMENTS ON BOTH UNMODIFIED AND MODIFIED VERSIONS}

The Open Publication works may be reproduced and distributed in whole or
in part, in any medium physical or electronic, provided that the terms
of this license are adhered to, and that this license or an
incorporation of it by reference (with any options elected by the
author(s) and/or publisher) is displayed in the reproduction.

Proper form for an incorporation by reference is as follows:

Copyright (c) $<$year$>$ by $<$author's name or designee$>$.  This material
may be distributed only subject to the terms and conditions set forth in
the Open Publication License, vX.Y or later (the latest version is
presently available at http://www.opencontent.org/openpub/).

The reference must be immediately followed with any options elected by
the author(s) and/or publisher of the document (see section VI).

Commercial redistribution of Open Publication-licensed material is
permitted.

Any publication in standard (paper) book form shall require the citation
of the original publisher and author. The publisher and author's names
shall appear on all outer surfaces of the book. On all outer surfaces of
the book the original publisher's name shall be as large as the title of
the work and cited as possessive with respect to the title.

{\bf II. COPYRIGHT}

The copyright to each Open Publication is owned by its author(s) or
designee.

{\bf III. SCOPE OF LICENSE}

The following license terms apply to all Open Publication works, unless
otherwise explicitly stated in the document.

Mere aggregation of Open Publication works or a portion of an Open
Publication work with other works or programs on the same media shall
not cause this license to apply to those other works. The aggregate work
shall contain a notice specifying the inclusion of the Open Publication
material and appropriate copyright notice.

SEVERABILITY. If any part of this license is found to be unenforceable
in any jurisdiction, the remaining portions of the license remain in
force.

NO WARRANTY. Open Publication works are licensed and provided "as is"
without warranty of any kind, express or implied, including, but not
limited to, the implied warranties of merchantability and fitness for a
particular purpose or a warranty of non-infringement.

{\bf IV. REQUIREMENTS ON MODIFIED WORKS}

All modified versions of documents covered by this license, including
translations, anthologies, compilations and partial documents, must meet
the following requirements:

\begin{enumerate}
 \item The modified version must be labeled as such.

 \item The person making the modifications must be identified and the
 modifications dated.

 \item Acknowledgement of the original author and publisher if
 applicable must be retained according to normal academic citation
 practices.

 \item The location of the original unmodified document must be
 identified.

 \item The original author's (or authors') name(s) may not be used to
 assert or imply endorsement of the resulting document without the
 original author's (or authors') permission.
\end{enumerate}

{\bf V. GOOD-PRACTICE RECOMMENDATIONS}

In addition to the requirements of this license, it is requested from
and strongly recommended of redistributors that:

\begin{enumerate} 

 \item If you are distributing Open Publication works on hardcopy or
 CD-ROM, you provide email notification to the authors of your intent to
 redistribute at least thirty days before your manuscript or media
 freeze, to give the authors time to provide updated documents.  This
 notification should describe modifications, if any, made to the
 document.

 \item All substantive modifications (including deletions) be either
 clearly marked up in the document or else described in an attachment to
 the document.
\end{enumerate}

Finally, while it is not mandatory under this license, it is considered
good form to offer a free copy of any hardcopy and CD-ROM expression of
an Open Publication-licensed work to its author(s).

{\bf VI. LICENSE OPTIONS}

The author(s) and/or publisher of an Open Publication-licensed document
may elect certain options by appending language to the reference to or
copy of the license. These options are considered part of the license
instance and must be included with the license (or its incorporation by
reference) in derived works.

A. To prohibit distribution of substantively modified versions without
the explicit permission of the author(s). "Substantive modification" is
defined as a change to the semantic content of the document, and
excludes mere changes in format or typographical corrections.

To accomplish this, add the phrase `Distribution of substantively
modified versions of this document is prohibited without the explicit
permission of the copyright holder.' to the license reference or copy.

B. To prohibit any publication of this work or derivative works in whole
or in part in standard (paper) book form for commercial purposes is
prohibited unless prior permission is obtained from the copyright
holder.

To accomplish this, add the phrase 'Distribution of the work or
derivative of the work in any standard (paper) book form is prohibited
unless prior permission is obtained from the copyright holder.' to the
license reference or copy.

{\bf OPEN PUBLICATION POLICY APPENDIX:}

(This is not considered part of the license.)

Open Publication works are available in source format via the Open
Publication home page at http://works.opencontent.org/.

Open Publication authors who want to include their own license on Open
Publication works may do so, as long as their terms are not more
restrictive than the Open Publication license.

If you have questions about the Open Publication License, please contact
TBD, and/or the Open Publication Authors' List at opal@opencontent.org,
via email.

\vspace*{\fill}

\newpage


\end{document}



