% This line is a comment because % precedes it.
% It specifies the format of head named 'title'
% Specifically the style of the font: sans serif
% + bold + big font.
 
\setuphead[title][style={\ss\bfd},
    before={\begingroup},
    after={John Doe, the author\smallskip\currentdate\bigskip\endgroup}]

\starttext 

 
\title{\CONTEXT}



\chapter{1 Option Stacking}

\section{Options}

\section{Option instances}

\chapter{2 Services}




\section{Options}



With some text that fits in here.

\section{twenty}



\chapter{3 Status of a subscription/option}
 
\chapter{Chapter 1}
\section{Textsss}

\CONTEXT\ is a document preparation system for the \TEX\ typesetting
program. It offers programmable desktop publishing features and extensive
facilities for automating most aspects of typesetting and desktop
publishing, including numbering and cross-referencing (for example
to equation \in[eqn:famous-emc]), tables and figures, page layout,
bibliographies, and much more.
 
It was originally written around 1990 by Hans Hagen. It could be an
alternative or complement to \LATEX.

\chapter{Chapter 2} 
\section{Maths}

With \CONTEXT\ we could write maths. Equations can be automatically numbered.

\placeformula[eqn:famous-emc]
\startformula
    E = mc^2
\stopformula
with
\placeformula[eqn:def-m]
\startformula
    m = \frac{m_0}{\sqrt{1-\frac{v^2}{c^2}}}
\stopformula


\stoptext
