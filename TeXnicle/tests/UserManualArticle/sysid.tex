\subsection{System identification along the $x$-axis}

As already stated above, one of the key steps in building up an accurate physical
model of LPF is the characterisation of the dynamical behaviour of the full system
along the most sensitive axis, the $x$-axis. It is important to note that the
dynamical time-constants involved in the system are of the same scale of the length
of an individual experiment (around 1 day). This means that we are analysing
a system that is never in a steady-state. As a result, we try, whenever possible,
to use time-domain approaches to ensure the transient behaviour of the system
is properly accounted for.

\cite{}

The characterisation of the dynamical system along the $x$-axis is done using a model
which contains a number of physical parameters. Each of these parameters must be
identified, either on ground or in flight. 

When operating in the main science mode, the residual motion of the bodies is minimised
due to the closed-loop system. As such, if we treat the system close to this
operating point, we can produce a linear parameterisation
of the components of the system equation shown above (Equation \ref{eq:sysmodel})
in the following way. First, the dynamics of the system can be described
by the following $2\times2$ Laplace-domain matrix
\begin{eqnarray}
\label{eqn:freedynamics} \mathbf{D^{-1}} &=& \left[ {\begin{array}{*{20}c}
   {s^2  + \omega _1^2  + \frac{m_1}{m_{\rm SC}}\omega _1^2  + \frac{m_2}{m_{\rm SC}}\omega
_2^2 } & {\frac{m_2}{m_{\rm SC}}\omega _2^2  + \Gamma _x }  \\    {\omega _2^2
 - \omega _1^2 } & {s^2  + \omega _2^2  - 2\Gamma _x }  \\ \end{array}} \right],\\\nonumber
\end{eqnarray} 
where $m_1$, $m_2$ and $m_{\rm sc}$ are the masses
of the TMs and SC respectively, the total stiffness coupling of each TM to the
SC is represented by the parameters $\omega_1^2$ and $\omega_2^2$, and $\Gamma _x$
represents the gravitational coupling of the two test-masses.  The sensing matrix,
$\mathbf{S}$ can be represented by 
\begin{eqnarray}   \mathbf{S} &=& \left[\begin{array}{cc}
  S_{11} & S_{12} \\   S_{21} & S_{22} \\ \end{array}\right].
\end{eqnarray} 
The control chain can be modelled as known controller functions
together with an unknown input delay due to the transmission of the signals from
the interferometer/DMU output along the spacecraft bus to the computer running
the control logic. In matrix form, this looks like 
\begin{eqnarray}   
\mathbf{C}
&=& \left[\begin{array}{cc}   C_{\rm df}e^{-sD_{1}} & 0 \\   0 & C_{\rm sus}e^{-sD_{12}}
 \\ \end{array}\right]. 
\end{eqnarray} 
Finally, the actuator is modelled
with an unknown gain and a response time characterised by a single pole, as follows
\begin{eqnarray}   
\mathbf{A} &=& \left[\begin{array}{cc}   \frac{A_{\rm
df}}{1 + s\tau_{\rm thrust}} & 0 \\   0 & \frac{A_{\rm sus}}{1 + s\tau_{\rm electro}}
 \\ \end{array}\right]. 
\end{eqnarray}

Table \ref{tab:params} lists the parameters that our model contains together with the predicted values.
\begin{table}[htdp]
\begin{center}
  \small
\begin{tabular}{|c|c|p{8cm}|} \hline
	\rowcolor[rgb]{0.8,0.8,0.8} Parameter & Typical Value & Description \\ \hline \hline
    $A_{\rm df}$  & 1.0 & The unknown gain of the thrusters. \\ \hline
    $A_{\rm sus}$ & 1.0 & The unknown gain of the electrostatic actuators. \\ \hline
    $S_{11}$  & 1.0 & The calibration of the interferometer which
senses the $x$-axis position of the SC relative to first TM ($X_1$). \\ \hline 
    $S_{12}$ & 0.0 & The cross-contamination from the $X_{12}$
interferometer to the $X_{1}$ interferometer. This represents the sensing of the motion of TM2 by the $X_1$ interferometer; it is expected to be negligible. \\ \hline
    $S_{21}$  & $1\times10^{-4}$ & The cross-contamination from the $X_{1}$ interferometer
to the $X_{12}$ interferometer. This represents the imperfect common-mode
rejection of the differential interferometer. \\ \hline
    $S_{22}$ & 1.0 & The calibration of the differential interferometer
($X_{12}$). \\ \hline
    $\omega_1^2$  & $-1\times10^{-6}$\,${\rm s}^{-2}$ & The total stiffness of TM 1 to the spacecraft. \\ \hline
    $\omega_{\Delta}^2$  & $-1\times10^{-6}$\,${\rm s}^{-2}$ & The difference of stiffness 
of the two TMs to the SC, defined as $\omega_{2}^2-\omega_1^2$. \\ \hline
    $\taudf$ & 0.1 s& The characteristic time of the thrusters. \\ \hline
    $\tausus$ & 0.01 s& The characteristic time of the electrostatic actuators. \\ \hline
    $D_{1}$  & 0.1 s& The bus delay between the $X_1$ interferometer output and the corresponding DFACS controller input. \\ \hline
    $D_{12}$  & 0.1 s& The bus delay between the $X_{12}$ interferometer output and the corresponding DFACS controller input. \\ \hline
\end{tabular}
\end{center}
\caption{The main physical parameters included in our model of the dynamical system 
of LPF along the $x$-axis.}
\label{tab:params}
\end{table}
In the following section, a pair of experiments is proposed that should lead to
the identification of the parameters given in Table \ref{tab:params}. 

%%MARK Experiments
\input{./xaxis_experiments.tex}





